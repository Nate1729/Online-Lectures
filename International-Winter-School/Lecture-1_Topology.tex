\documentclass{article}

\usepackage{fancyhdr}
\usepackage{amssymb}
\usepackage{amsthm}
\usepackage{amsmath}
\usepackage{mathrsfs}
\usepackage[margin=1in]{geometry}
\newtheorem*{definition_star}{Definition}
\newtheorem{definition}{Definition}
\renewcommand{\labelenumi}{(\roman{enumi})}
\pagestyle{fancy}
\fancyhf{}
\lhead{Lecture 1: Topology}
\rhead{Winter School on Gravity and Light}


\begin{document}
	\begin{definition_star}[Spacetime]
		Spacetime is a four-dimensional topological manifold with a smooth
		atlas carrying a torsion-free connection compatible with a Lorenzian
		metric and a time orientation satisying the Einstein equations.
	\end{definition_star}
	The purpose of these lectures is to clarify this definition.

	\section{Toplogy}
	\noindent At the coarsest level, spacetime is a set. However, this amount of structure is
	not enough to discuss concepts that are important to physics. Specifically, one
	cannot discuss \textit{continuity} with only the strucutre of a set.

	\noindent Why do we care about continuity? We care about continuity because in classical
	physics ``curves to not jump.'' The trajectory of a particle does not jump around
	in space.

	\noindent How does one decide to implement a notion of continuity? Well a good place to 
	start is to try and be as ``economical'' as possible. That is to say use the
	weakest structure possible that captures the idea. We do this because we do not 
	want to introduce unneccesary assumptions. In mathematics the weakest structure
	which gives a notion of continuity is a ``topology.''

	\begin{definition}
		Let $M$ be a set. A Topology $O$ is a subset of $O \subseteq \mathscr{P}(M)$
		satisying,
		\begin{enumerate}
			\item $\emptyset \in O$, $M\in O$
			\item $\forall U,V\in O \Rightarrow U\cap V\in O$
			\item $\forall \lambda\in\Lambda : U_\lambda\in O\Rightarrow 
			\cup_{\lambda\in\Lambda} U_\lambda \:\in O$
		\end{enumerate}
	\end{definition}
	\noindent The mneumonic people will sometimes use to remember this is that arbitrary unions
	and finite intersections close in the topology.

	\noindent\textit{Example}. $M = \{1,2,3\}$

	\noindent $O_1:=\{\emptyset$, $\{1,2,3\}\}$. This is a topology on $M$.

	\noindent $O_2:=\{\emptyset$, $\{1\}$, $\{2\}$, $\{1,2,3\}\}$ This is not a topology
	because $\{1\}\cup\{2\}$ is not in $O_2$.
	\vspace{0.25cm}	

	\noindent There are two topologies that one can also create for any set $M$.

	\noindent\textit{The Chaotic Topology} \\
	$ O_{\textrm{chaotic}} := \{\emptyset, M\}$
	
	\noindent\textit{The Discrete Topology}\\
	$ O_{\textrm{discrete}} := \mathscr{P}(M) $

	\noindent These are extreme examples of topologies. These topologies show the maximal
	and minimal cases, in terms of sizes of topology. In practice they are completely
	useless. The reason for this will be clear after the definition of continuity.

	\noindent\textit{Example} The standard topology. 
	The standard topology must be defined on a set $M =\mathscr{R}^d$. Note that there 
	are uncountable many elements in the standard topology, which means unlike
	the previous examples, we cannot enumerate each element.

	\noindent We construct the standard topology by first defining a soft ball $B_r$.
	$$ B_r(p) := \{(q_1,...,q_d) | \sum_{i=1}^{d} (q_i-p_i)^2 < r^2\} $$

	\noindent Now we say $U\in O_{standard}$ iff
	$$ \forall p\in U: \exists r\in \mathscr{R}^+ : B_r(p) \subseteq U $$
	\vspace{0.25cm}
	\noindent\textit{Terminology}
	Let $M$ be a set using (ZFC) Zermelo Frankel set theory with axiom of choice.\\
	We call $O$ a topology.\\
	$(M, O)$ is a topological space.\\
	$U\in O$ we call $U$ an open set.\\
	$M\textrm{\textbackslash} A\in O$ we call $A$ a closed set.\\
	Note that being open does not imply anything about being closed.

	\section{Continuous Maps}
	\noindent A quick review of maps. A map $f:M\longrightarrow N$ where $M$ and $N$ are both 
	sets and are refered to as the domain and target (co-domain) respectively. An 
	important observation is that $f$ maps \textit{everything} in its domain. However,
	this requirement does not apply to the target. In fact, the case where $f$ "hits"
	every point in the target is a special type of map known as a surjection. Another
	special type of map is where each point of the domain is mapped uniquely. Such
	a map is called an injection or sometimes called one-to-one. It should be clear
	that these types of maps are independent of each other. Which is why the case
	where a map has both of these properties is given another name, bijection.

	\noindent What if we asked, is $f$ continuous? The answer is: it depends. What it
	depends on is the choice of toplogy on \textit{both} the set $M$ and $N$.
	\begin{definition}
		Let $(M,\mathscr{O}_M)$ and $(N,\mathscr{O}_N)$ be topological spaces. Then a 
		map $f:M\longrightarrow N$ is called continuous (with respect to $\mathscr{O}_M)$
		and $\mathscr{O}_N$) if
		$$ \forall V\in \mathscr{O}_N : preim_f(V) \in \mathscr{O}_M$$
	\end{definition}
	\noindent A quick aside for what $preim$ means. Take a map $f:M\longrightarrow N$ and look
	at a set $V\subseteq N$. We have the following definition,
	$$ preim_f(V) := \{m\in M | f(m) \in V\}$$
	It is tempting to say that $preim$ is just the inverse of a map. However,
	this is not the case. The preimage of a map is a more general notion of inverse.

	\noindent Useful mnemonic: A map is continuous if the preimages of open sets
	are open sets.

	\noindent\textit{Example} Let $M:=\{1,2\}$ and $N:=\{1,2\}$ and define
	$f:M\longrightarrow N$ as:
	$$ f(m):=
	\begin{cases}
		1 & m=2 \\
		2 & m=1 
	\end{cases}
	$$
	For the toplogies we pick $\mathscr{O}_M:=\{\emptyset, \{1\}, \{2\}, \{1,2\}\}$ 
	and $\mathscr{O}_N:=\{\emptyset, \{1,2\}\}$. We explicitly check:
	$$ preim_f(\emptyset) = \emptyset\in\mathscr{O}_M $$ 
	$$ preim_f(\{1,2\}) = M \in\mathscr{O}_M $$
	What about the map $g:N\longrightarrow M$, is this map continuous. Well
	let's check,
	$$ preim_f(\{1\}) = \{2\}\notin \mathscr{O}_N$$
	Therefore $g$ is \textbf{not} continuous.

	\noindent From now-on unless explicitly stated otherwise we assume the standard
	toplogy for $\mathscr{R}^d$

	\section{Composition of Continuous Maps}

	
	\end{document}